\documentclass{article}
\usepackage[a4paper, total={6in, 8in}]{geometry}
\title{MCOT - Entrainer des réseaux de neuronnes à la résolution d'équation aux dérivés partielles: application à la prevention de rupture d'anévirsme cérébrale}

\begin{document}
	\maketitle
	
	\section{Motivation}	
	
	Je m'intéresse depuis quelques temps à l'intéligence artificielle, et notamment au Deep Lerning avec les réseaux de neuronnes artificielles. J'ai également développé dans ma première année de CPGE une appétence pour la physique, j'ai donc voulu combiner les deux dans ce projet de TIPE
	
	\section{Ancrage}

	\section{Positionement thématique et mots clés}
	
	\section{Bibliographie commentée}	
	
	\section{Problématique}	
	
	\section{Objectifs}
	
	\section{Références bibliographiques}
	
	\begin{thebibliography}{2} % 100 is a random guess of the total number of

	\bibitem[1]{Maziar Raissi1, Paris Perdikaris2, George Em Karniadakis1} Physics Informed Deep Learning (Part I): Data-driven
Solutions of Nonlinear Partial Differential Equations \emph{https://arxiv.org/pdf/1711.10561}

\end{thebibliography}
\end{document}